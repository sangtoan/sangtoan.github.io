
\de{ĐỀ THI HỌC KỲ II NĂM HỌC 2022-2023}{THPT SỐ 2 BẢO THẮNG - Đồng Nai}
\begin{center}
	\textbf{PHẦN 1 - TRẮC NGHIỆM}
\end{center}
\Opensolutionfile{ans}[ans/ans]
%Câu 1...........................
\begin{ex}%[0D2Y1-1]%[Dự án đề kiểm tra HKII NH22-23- Nguyễn Cường]%[THPT Bảo Thắng Số 2-Lào Cai]
Bạn A đi mua một cái áo thu ở một cửa hàng. Giả sử cửa hàng có $10$ cái áo thun size S, $5$ cái áo thun size M, $18$ áo size L. Giả sử áo size nào bạn A cũng mang được. Hỏi bạn A có bao nhiêu cách lựa chọn?
	\choice
	{\True $33$ áo}
	{$900$ áo}
	{$15$ áo}
	{$23$ áo}
	\loigiai{
	Bạn A có $10+5+18=33$ cách chọn áo.
	}
\end{ex}
%Câu 2...........................
\begin{ex}%[0D2Y1-2]%[Dự án đề kiểm tra HKII NH22-23- Nguyễn Cường]%[THPT Bảo Thắng Số 2-Lào Cai]
	Từ thành phố A đến thành phố B có $6$ con đường, từ thành phố B đến thành phố C có $7$ con đường. Có bao nhiêu cách đi từ thành phố A đến thành phố C, biết phải đi qua thành phố B?
	\choice
	{\True $42$}
	{$46$}
	{$48$}
	{$44$}
	\loigiai{
	\begin{itemize}
		\item Đi từ thành phố A đến thành phố B có $6$ cách.
		\item Từ thành phố B đến thành phố C có $7$ cách.
	\end{itemize}
Theo quy tắc nhân ta có $7\cdot 6=42$ cách đi từ thành phố A đến thành phố C, biết phải đi qua thành phố B.
	}
\end{ex}
%Câu 3...........................
\begin{ex}%[0D2Y1-2]%[Dự án đề kiểm tra HKII NH22-23- Nguyễn Cường]%[THPT Bảo Thắng Số 2-Lào Cai]
	Một người vào cửa hàng ăn, người đó chọn thực đơn gồm $1$ món ăn trong $5$ món, $1$ loại quả tráng miệng trong $5$ loại quả tráng miệng và một nước uống trong $3$ loại nước uống. Có bao nhiêu cách chọn thực đơn?
	\choice
	{$25$}
	{\True $75$}
	{$100$}
	{$15$}
	\loigiai{
		\begin{itemize}
			\item Chọn $1$ món ăn có $5$ cách.
			\item Chọn $1$ quả tráng miệng có $5$ quả.
			\item Chọn $1$ loại nước uống có $3$ cách.
		\end{itemize}
		Theo quy tắc nhân ta có $5\cdot 5\cdot 3=75$ cách chọn thực đơn.
	}
\end{ex}
%Câu 4...........................
\begin{ex}%[0D2Y1-1]%[Dự án đề kiểm tra HKII NH22-23- Nguyễn Cường]%[THPT Bảo Thắng Số 2-Lào Cai]
	Lớp 12A có $43$ học sinh, lớp 12B có $30$ học sinh. Hỏi có bao nhiêu cách chọn $1$ học sinh từ lớp 12A và 12B?
	\choice
	{$43$}
	{$30$}
	{\True $73$}
	{$1290$}
	\loigiai{
	Chọn $1$ học sinh từ lớp 12A và 12B có $43+30=73$ cách.
	}
\end{ex}
%Câu 5...........................
\begin{ex}%[0D2Y1-2]%[Dự án đề kiểm tra HKII NH22-23- Nguyễn Cường]%[THPT Bảo Thắng Số 2-Lào Cai]
	Cho tập $A=\left\{0;1;2;3;4;5\right\}$. Có bao nhiêu số tự nhiên lẻ gồm $2$ chữ số được lập từ tập $A$?
	\choice
	{$9$}
	{$12$}
	{\True $15$}
	{$18$}
	\loigiai{
	Gọi $\overline{ab}$ là số cần lập.
	\begin{itemize}
		\item Chọn chữ số $b$ có $3$ cách do $b\in\{1;3;5\}$.
		\item Chọn chữ số $a$ có $5$ cách do $a\ne 0$.
	\end{itemize}
Do đó, có $5\cdot 3=15$ số được lập.
	}
\end{ex}
%Câu 6...........................
\begin{ex}%[0D2Y1-1]%[Dự án đề kiểm tra HKII NH22-23- Nguyễn Cường]%[THPT Bảo Thắng Số 2-Lào Cai]
	Giả sử từ tỉnh A đến tỉnh B có thể đi bằng các phương tiện: ô tô, tàu hỏa, tàu thủy hoặc máy bay.
	Mỗi ngày có $10$ chuyến ô tô, $5$ chuyến tàu hỏa, $3$ chuyến tàu thủy và $2$ chuyến máy bay. Hỏi có bao nhiêu cách đi từ tỉnh A đến tỉnh B?
	\choice
	{$20$}
	{$300$}
	{\True $18$}
	{$15$}
	\loigiai{
	Đi từ tỉnh A đến tỉnh B có $10+5+3=18$ cách.
	}
\end{ex}
%Câu 7...........................
\begin{ex}%[0D2Y2-1]%[Dự án đề kiểm tra HKII NH22-23- Nguyễn Cường]%[THPT Bảo Thắng Số 2-Lào Cai]
	 Có bao nhiêu cách xếp $4$ học sinh vào bốn chiếc ghế được xếp thành hàng ngang?
	\choice
	{\True $4!$}
	{$4^4$}
	{$16$}
	{$10$}
	\loigiai{
	Xếp $4$ học sinh vào bốn chiếc ghế có $4!$ cách.
	}
\end{ex}
%Câu 8...........................
\begin{ex}%[0D2Y2-2]%[Dự án đề kiểm tra HKII NH22-23- Nguyễn Cường]%[THPT Bảo Thắng Số 2-Lào Cai]
	Tổ một có $11$ học sinh. Cô giáo muốn chọn $6$ bạn đi trực nhật khu tự quản. Hỏi cô giáo có bao nhiêu cách chọn?
	\choice
	{$\mathrm{A}_{11}^6$}
	{\True $\mathrm{C}_{11}^6$}
	{$16$}
	{$10$}
	\loigiai{
	Cô giáo có $\mathrm{C}_{11}^6$ cách chọn.
	}
\end{ex}
%Câu 9...........................
\begin{ex}%[0D2Y2-5]%[Dự án đề kiểm tra HKII NH22-23- Nguyễn Cường]%[THPT Bảo Thắng Số 2-Lào Cai]
	 Từ $7$ chữ số $1$, $2$, $3$, $4$, $5$, $6$, $7$ có thể lập được bao nhiêu số tự nhiên có $4$ chữ số khác nhau?
	\choice
	{\True $\mathrm{A}_{7}^4$}
	{$\mathrm{C}_{7}^4$}
	{$7!$}
	{$7^4$}
	\loigiai{
	Có $\mathrm{A}_{7}^4$ số tự nhiên thỏa mãn.
	}
\end{ex}
%Câu 10...........................
\begin{ex}%[0D2Y2-2]%[Dự án đề kiểm tra HKII NH22-23- Nguyễn Cường]%[THPT Bảo Thắng Số 2-Lào Cai]
 	Số các chỉnh hợp chập $k$ của $n$ phần tử được kí hiệu là
	\choice
	{\True $\mathrm{A}_{n}^k$}
	{$\mathrm{C}_{n}^k$}
	{$\mathrm{P}_n$}
	{$k!$}
	\loigiai{
		Số các chỉnh hợp chập $k$ của $n$ phần tử được kí hiệu là $\mathrm{A}_{n}^k$.
	}
\end{ex}
%Câu 11...........................
\begin{ex}%[0D2Y2-2]%[Dự án đề kiểm tra HKII NH22-23- Nguyễn Cường]%[THPT Bảo Thắng Số 2-Lào Cai]
	Số các tổ hợp chập $k$ của $n$ phần tử được tính bởi công thức
	\choice
	{$\mathrm{C}_{n}^k=\dfrac{n!}{k!}$}
	{$\mathrm{C}_{n}^k=\dfrac{n}{k(n-k)}$}
	{\True $\mathrm{C}_{n}^k=\dfrac{n!}{k!(n-k)!}$}
	{$\mathrm{C}_{n}^k=\dfrac{n!}{(n-k)!}$}
	\loigiai{
	Ta có $\mathrm{C}_{n}^k=\dfrac{n!}{k!(n-k)!}$.
	}
\end{ex}
%Câu 12...........................
\begin{ex}%[0D2B2-2]%[Dự án đề kiểm tra HKII NH22-23- Nguyễn Cường]%[THPT Bảo Thắng Số 2-Lào Cai]
	Hội đồng quản trị của công ty X gồm $10$ người. Hỏi có bao nhiêu cách bầu ra ba người vào ba vị trí chủ tịch, phó chủ tịch và thư kí, biết khả năng mỗi người là như nhau?
	\choice
	{$728$}
	{$723$}
	{\True $720$}
	{$722$}
	\loigiai{
		Mỗi cách bầu ra $3$ người trong $10$ người vào $3$ vị trí khác nhau là một chỉnh hợp chập $3$ của $10$ phần tử.\\
		Vậy có $\mathrm{A}_{10}^3=720$ cách.
	}
\end{ex}
%Câu 13...........................
\begin{ex}%[0D2B2-2]%[Dự án đề kiểm tra HKII NH22-23- Nguyễn Cường]%[THPT Bảo Thắng Số 2-Lào Cai]
	Tên $15$ học sinh được ghi vào $15$ tờ giấy để vào trong hộp. Chọn tên $4$ học sinh để cho đi du lịch.
	Hỏi có bao nhiêu cách chọn các học sinh?
	\choice
	{$4!$}
	{$15!$}
	{\True $1365$}
	{$32760$}
	\loigiai{
	Chọn $4$ học sinh trong $15$ học sinh có $\mathrm{C}_{15}^4=1365$ cách.
	}
\end{ex}
%Câu 14...........................
\begin{ex}%[0D2B2-1]%[Dự án đề kiểm tra HKII NH22-23- Nguyễn Cường]%[THPT Bảo Thắng Số 2-Lào Cai]
	Có bao nhiêu cách xếp bốn bạn Hà, Mai, Nam, Dương thành một hàng dọc biết bạn Mai đứng đầu hàng?
	\choice
	{$7$}
	{$24$}
	{\True $6$}
	{$12$}
	\loigiai{
	Bạn Mai đứng đầu hàng nên có $1$ cách xếp bạn Mai đứng vào hàng.\\
	Xếp $3$ bạn còn lại vào $3$ vị trí có $3!=6$ cách.\\
	Vậy có $6$ cách xếp chỗ thỏa mãn yêu cầu.
	}
\end{ex}
%Câu 15...........................
\begin{ex}%[0D2B2-2]%[Dự án đề kiểm tra HKII NH22-23- Nguyễn Cường]%[THPT Bảo Thắng Số 2-Lào Cai]
	Trong một buổi khiêu vũ có $20$ nam và $18$ nữ. Hỏi có bao nhiêu cách chọn ra một đôi nam nữ để	khiêu vũ?
	\choice
	{$\mathrm{C}_{38}^2$}
	{$\mathrm{A}_{38}^2$}
	{$\mathrm{C}_{20}^2\cdot \mathrm{C}_{18}^1$}
	{\True $\mathrm{C}_{20}^1\cdot \mathrm{C}_{18}^1$}
	\loigiai{
	Chọn $1$ nữ trong $18$ nữ có $\mathrm{C}_{18}^1$ cách.\\
	Chọn $1$ nam trong $20$ nam có $\mathrm{C}_{20}^1$ cách.\\
	Vậy có $\mathrm{C}_{20}^1\cdot \mathrm{C}_{18}^1$ cách chọn một đôi nam nữ để khiêu vũ.
	}
\end{ex}
%Câu 16...........................
\begin{ex}%[0D2Y3-1]%[Dự án đề kiểm tra HKII NH22-23- Nguyễn Cường]%[THPT Bảo Thắng Số 2-Lào Cai]
	Trong khai triển $(2x-5y)^4$ có bao nhiêu số hạng?
	\choice
	{$6$}
	{$4$}
	{$7$}
	{\True $5$}
	\loigiai{
		Trong khai triển $(2x-5y)^4$ có $5$ số hạng.
	}
\end{ex}
%Câu 17...........................
\begin{ex}%[0D2Y3-1]%[Dự án đề kiểm tra HKII NH22-23- Nguyễn Cường]%[THPT Bảo Thắng Số 2-Lào Cai]
	Trong khai triển $(a+b)^5$, hệ số của số hạng thứ ba theo thứ tự giảm bậc dần của $a$ bằng
	\choice
	{$\mathrm{C}_5^4$}
	{\True $\mathrm{C}_5^2$}
	{$\mathrm{C}_6^3$}
	{$\mathrm{C}_6^2$}
	\loigiai{
		Ta có số hạng thứ $3$ trong khai triển theo thứ tự giảm dần bậc của $a$ là $\mathrm{C}_5^2\cdot a^3b^2$.\\
		Do đó, hệ số cần tìm là $\mathrm{C}_5^2$.
	}
\end{ex}
%Câu 18...........................
\begin{ex}%[0D2Y3-1]%[Dự án đề kiểm tra HKII NH22-23- Nguyễn Cường]%[THPT Bảo Thắng Số 2-Lào Cai]
	Khai triển $(a+2)^4$ ta được kết quả đúng là
	\choice
	{\True $\mathrm{C}_4^0\cdot a^4+\mathrm{C}_4^1\cdot a^3\cdot 2+\mathrm{C}_4^2\cdot a^2\cdot 2^2+\mathrm{C}_4^3\cdot a\cdot 2^3+\mathrm{C}_4^4\cdot 2^4$}
	{$\mathrm{C}_4^0\cdot a^4+\mathrm{C}_4^1\cdot a^3\cdot 2-\mathrm{C}_4^2\cdot a^2\cdot 2^2+\mathrm{C}_4^3\cdot a\cdot 2^3+\mathrm{C}_4^4\cdot 2^4$}
	{$\mathrm{C}_4^1\cdot a^3\cdot 2+\mathrm{C}_4^2\cdot a^2\cdot 2^2+\mathrm{C}_4^3\cdot a\cdot 2^3+\mathrm{C}_4^4\cdot 2^4$}
	{$\mathrm{C}_4^0\cdot a^4+\mathrm{C}_4^1\cdot a^3\cdot 2-\mathrm{C}_4^2\cdot a^2\cdot 2^2+\mathrm{C}_4^3\cdot a\cdot 2^3-\mathrm{C}_4^4\cdot 2^4$}
	\loigiai{
		Ta có $(a+2)^4=\mathrm{C}_4^0\cdot a^4+\mathrm{C}_4^1\cdot a^3\cdot 2+\mathrm{C}_4^2\cdot a^2\cdot 2^2+\mathrm{C}_4^3\cdot a\cdot 2^3+\mathrm{C}_4^4\cdot 2^4$.
	}
\end{ex}
%Câu 19...........................
\begin{ex}%[0D2B3-1]%[Dự án đề kiểm tra HKII NH22-23- Nguyễn Cường]%[THPT Bảo Thắng Số 2-Lào Cai]
	Trong khai triển $(0{,}3+0{,}7)^4$, hệ số của số hạng thứ hai là
	\choice
	{\True $0{,}0756$}
	{$0{,}2646$}
	{$0{,}3752$}
	{$0{,}2463$}
	\loigiai{
	Ta có
	\allowdisplaybreaks
	\begin{eqnarray*}
		(0{,}3+0{,}7)^4&=&\mathrm{C}_4^0\cdot 0{,}3^4+\mathrm{C}_4^1\cdot 0{,}3^3\cdot 0{,}7+\mathrm{C}_4^2\cdot 0{,}3^2\cdot 0{,}7^2+\mathrm{C}_4^3\cdot 0{,}3\cdot 0{,}7^3+\mathrm{C}_4^4\cdot 0{,}7^4\\
		&=&0{,}0081+0{,}0756+0{,}2646+0{,}4116+0{,}2401.
	\end{eqnarray*}
Do đó, số hạng thứ hai trong khai triển là $0{,}0756$.
	}
\end{ex}
%Câu 20...........................
\begin{ex}%[0D2B3-1]%[Dự án đề kiểm tra HKII NH22-23- Nguyễn Cường]%[THPT Bảo Thắng Số 2-Lào Cai]
	Trong khai triển $(x-2)^{n+2}$ với $n\in\mathbb{N}$ có tất cả $5$ số hạng. Vậy $n$ là
	\choice
	{$3$}
	{\True $2$}
	{$6$}
	{$5$}
	\loigiai{
	Ta có $n+2+1=5\Leftrightarrow n=2$.
	}
\end{ex}
%Câu 21...........................
\begin{ex}%[0D3Y1-1]%[Dự án đề kiểm tra HKII NH22-23- Nguyễn Cường]%[THPT Bảo Thắng Số 2-Lào Cai]
	Gieo một con súc sắc, số phần tử của không gian mẫu là
	\choice
	{$24$}
	{$12$}
	{$8$}
	{\True $6$}
	\loigiai{
		Ta có $n(\Omega)=6$.
	}
\end{ex}
%Câu 22...........................
\begin{ex}%[0D3Y1-1]%[Dự án đề kiểm tra HKII NH22-23- Nguyễn Cường]%[THPT Bảo Thắng Số 2-Lào Cai]
	Gieo $2$ đồng tiền là một phép thử ngẫu nhiên có không gian mẫu là
	\choice
	{\True $\{NN,NS,SN,SS\}$}
	{$\{NS,SN\}$}
	{$\{N,S\}$}
	{$\{NNN,SSS,NNS,SSN,NSS,SNN\}$}
	\loigiai{
	Ta có không gian mẫu là $\Omega=\{NN,NS,SN,SS\}$.
	}
\end{ex}
%Câu 23...........................
\begin{ex}%[0D3Y1-2]%[Dự án đề kiểm tra HKII NH22-23- Nguyễn Cường]%[THPT Bảo Thắng Số 2-Lào Cai]
	Gieo con súc sắc $2$ lần. Biến cố nào dưới đây là biến cố tổng số chấm trong hai lần gieo bằng $7$?
	\choice
	{\True $A=\left\{(1;6),(6;1),(3;4),(4;3),(5;2),(2;5)\right\}$}
	{$B=\left\{(1;6),(3;4),(5;2)\right\}$}
	{$C=\left\{(1;6),(6;1),(3;4),(4;3)\right\}$}
	{$D=\left\{(2;6),(6;2),(3;4),(4;3),(5;2),(2;5)\right\}$}
	\loigiai{
	Ta có $A=\left\{(1;6),(6;1),(3;4),(4;3),(5;2),(2;5)\right\}$.
	}
\end{ex}
%Câu 24...........................
\begin{ex}%[0D3B1-2]%[Dự án đề kiểm tra HKII NH22-23- Nguyễn Cường]%[THPT Bảo Thắng Số 2-Lào Cai]
	Rút ngẫu nhiên một hộp chứa $9$ tấm thẻ được đánh số $1$; $2$; $3$; $4$; $5$; $6$; $7$; $8$; $9$. Gọi $A$ là biến cố: “Rút được thẻ ghi số chia hết cho $3$”. Biến cố đối của biến cố $A$ là
	\choice
	{$\overline{A}=\left\{1;2;4;5;7;8\right\}$}
	{$\overline{A}=\left\{3;6;9\right\}$}
	{\True $\overline{A}=\left\{1;2;4;5;7;8;10\right\}$}
	{$\overline{A}=\left\{2;4;6;8\right\}$}
	\loigiai{
		Biến cố đối của biến cố $A$ là “Rút được thẻ ghi số không chia hết cho $3$”. \\
		Do đó, $\overline{A}=\left\{1;2;4;5;7;8;10\right\}$.
	}
\end{ex}
%Câu 25...........................
\begin{ex}%[0D3Y2-1]%[Dự án đề kiểm tra HKII NH22-23- Nguyễn Cường]%[THPT Bảo Thắng Số 2-Lào Cai]
	Cho phép thử $T$ có không gian mẫu là $\Omega$ . Gọi $A$ là một biến cố liên quan đến phép thử $T$. Giả sử $n(A)=4$; $n(\Omega)=13$ thì xác suất của biến cố $A$ bằng?
	\choice
	{$\dfrac{13}{4}$}
	{\True $\dfrac{4}{13}$}
	{$4\cdot 13$}
	{$13-4$}
	\loigiai{
		Xác suất của biến cố $A$ là $\mathrm{P}(A)=\dfrac{n(A)}{n(\Omega)}=\dfrac{4}{13}$.
	}
\end{ex}
%Câu 26...........................
\begin{ex}%[0D3Y2-1]%[Dự án đề kiểm tra HKII NH22-23- Nguyễn Cường]%[THPT Bảo Thắng Số 2-Lào Cai]
	Gieo $2$ con súc sắc cân đối và đồng chất. Xác suất để tổng số chấm xuất hiện trên hai mặt của $2$
	con súc sắc đó không vượt quá $5$ là
	\choice
	{$\dfrac{2}{34}$}
	{$\dfrac{7}{18}$}
	{$\dfrac{8}{9}$}
	{\True $\dfrac{5}{18}$}
	\loigiai{
	Ta có $n(\Omega)=36$.\\
	Biến cố $A$ \lq\lq Tổng số chấm xuất hiện trên hai mặt của hai con súc sắc không quá $5$\rq\rq.\\
	$A=\left\{(1;1),(1;2);(1;3);(1;4);(2;1);(2;2);(2;3);(3;1);(3;2);(4;1)\right\}$, suy ra $n(A)=10$.\\
	Vậy xác suất biến cố $A$ là $\mathrm{P}(A)=\dfrac{n(A)}{n(\Omega)}=\dfrac{10}{36}=\dfrac{5}{18}$.
	}
\end{ex}
%Câu 27...........................
\begin{ex}%%[0D3B2-1]%[Dự án đề kiểm tra HKII NH22-23- Nguyễn Cường]%[THPT Bảo Thắng Số 2-Lào Cai]
	Gieo một đồng tiền liên tiếp $3$ lần. Xác suất của biến cố $A$: “Kết quả của $3$ lần gieo là như nhau” là
	\choice
	{$\dfrac{1}{2}$}
	{$\dfrac{3}{8}$}
	{$\dfrac{7}{8}$}
	{\True $\dfrac{1}{4}$}
	\loigiai{
		Ta có $n(\Omega)=8$.\\
		Biến cố $A$ \lq\lq Kết quả của $3$ lần gieo là như nhau\rq\rq.\\
		$A=\left\{NNN,SSS\right\}$, suy ra $n(A)=2$.\\
		Vậy xác suất biến cố $A$ là $\mathrm{P}(A)=\dfrac{n(A)}{n(\Omega)}=\dfrac{2}{8}=\dfrac{1}{4}$.
	}
\end{ex}
%Câu 28...........................
\begin{ex}%[0D3B2-6]%[Dự án đề kiểm tra HKII NH22-23- Nguyễn Cường]%[THPT Bảo Thắng Số 2-Lào Cai]
	Từ các chữ số $1$; $2$; $4$; $6$; $8$; $9$ lấy ngẫu nhiên một số. Xác suất để lấy được một số nguyên tố là
	\choice
	{$\dfrac{1}{2}$}
	{$\dfrac{1}{3}$}
	{$\dfrac{1}{4}$}
	{\True $\dfrac{1}{6}$}
	\loigiai{
		Ta có $n(\Omega)=6$.\\
		Biến cố $A$ \lq\lq Lấy được một số nguyên tố\rq\rq.\\
		$A=\left\{2\right\}$, suy ra $n(A)=1$.\\
		Vậy xác suất biến cố $A$ là $\mathrm{P}(A)=\dfrac{n(A)}{n(\Omega)}=\dfrac{1}{6}$.
	}
\end{ex}
%Câu 29...........................
\begin{ex}%[0D3Y2-9]%[Dự án đề kiểm tra HKII NH22-23- Nguyễn Cường]%[THPT Bảo Thắng Số 2-Lào Cai]
	Cho $A$ là một biến cố liên quan phép thử $T$. Mệnh đề nào sau đây là mệnh đề đúng ?
	\choice
	{$\mathrm{P}(A)$ là số lớn hơn $0$}
	{\True $\mathrm{P}(A)=1-\mathrm{P}(\overline{A})$}
	{$\mathrm{P}(A)=0\Leftrightarrow A=\Omega$}
	{$\mathrm{P}(A)$ là số nhỏ hơn $1$}
	\loigiai{
		$\mathrm{P}(A)=1-\mathrm{P}(\overline{A})$ là mệnh đề đúng.
	}
\end{ex}
%Câu 30...........................
\begin{ex}%[0D3Y1-1]%[Dự án đề kiểm tra HKII NH22-23- Nguyễn Cường]%[THPT Bảo Thắng Số 2-Lào Cai]
	Có $100$ tấm thẻ được đánh số từ $1$ đến $100$. Lấy ngẫu nhiên $5$ thẻ. Số phần tử của không gian mẫu
	là
	\choice
	{\True $n(\Omega)=\mathrm{C}_{100}^5$}
	{$n(\Omega)=\mathrm{A}_{100}^5$}
	{$n(\Omega)=\mathrm{C}_{100}^1$}
	{$n(\Omega)=\mathrm{A}_{100}^1$}
	\loigiai{
	Ta có $n(\Omega)=\mathrm{C}_{100}^5$.
	}
\end{ex}
%Câu 31...........................
\begin{ex}%[0D3Y1-1]%[Dự án đề kiểm tra HKII NH22-23- Nguyễn Cường]%[THPT Bảo Thắng Số 2-Lào Cai]
	Trong một chiếc hộp đựng $6$ viên bi đỏ, $8$ viên bi xanh, $10$ viên bi trắng. Lấy ngẫu nhiên $4$ viên bi.
	Số phần tử của không gian mẫu là?
	\choice
	{\True $n(\Omega)=\mathrm{C}_{24}^4$}
	{$n(\Omega)=\mathrm{A}_{24}^4$}
	{$n(\Omega)=\mathrm{C}_6^4+\mathrm{C}_8^4+\mathrm{C}_{10}^4$}
	{$n(\Omega)=6\cdot 8\cdot 10$}
	\loigiai{
	Ta có $n(\Omega)=\mathrm{C}_{24}^4$.
	}
\end{ex}
%Câu 32...........................
\begin{ex}%[0D3B2-3]%[Dự án đề kiểm tra HKII NH22-23- Nguyễn Cường]%[THPT Bảo Thắng Số 2-Lào Cai]
	Một tổ trong lớp 10A có $4$ bạn nam và $6$ bạn nữ. Giáo viên chọn ngẫu nhiên hai bạn trong tổ để tham gia đội làm báo của lớp. Xác suất để hai bạn được chọn có một bạn nam và một bạn nữ là?
	\choice
	{$\dfrac{2}{15}$}
	{$\dfrac{6}{25}$}
	{$\dfrac{8}{25}$}
	{\True $\dfrac{8}{15}$}
	\loigiai{
	Ta có $n(\Omega)=\mathrm{C}_{10}^2=45$.\\
	Biến cố $A$ \lq\lq Hai bạn được chọn có một bạn nam và một bạn nữ\rq\rq.\\
	Suy ra, $n(A)=\mathrm{C}_6^1\cdot\mathrm{C}_4^1=24$.\\
	Vậy xác suất biến cố $A$ là $\mathrm{P}(A)=\dfrac{n(A)}{n(\Omega)}=\dfrac{24}{45}=\dfrac{8}{15}$.
	}
\end{ex}
%Câu 33...........................
\begin{ex}%[0D3B2-4]%[Dự án đề kiểm tra HKII NH22-23- Nguyễn Cường]%[THPT Bảo Thắng Số 2-Lào Cai]
	Có $3$ viên bi đỏ và $7$ viên bi xanh, lấy ngẫu nhiên $4$ viên bi. Tính xác suất để lấy được $2$ viên bi đỏ và $2$ viên bi xanh.
	\choice
	{$\dfrac{7}{10}$}
	{\True $\dfrac{3}{10}$}
	{$\dfrac{4}{35}$}
	{$\dfrac{17}{35}$}
	\loigiai{
		Ta có $n(\Omega)=\mathrm{C}_{10}^4=210$.\\
		Biến cố $A$ \lq\lq Lấy được $2$ viên bi đỏ và $2$ viên bi xanh\rq\rq.\\
		Suy ra, $n(A)=\mathrm{C}_7^2\cdot\mathrm{C}_3^2=63$.\\
		Vậy xác suất biến cố $A$ là $\mathrm{P}(A)=\dfrac{n(A)}{n(\Omega)}=\dfrac{63}{210}=\dfrac{3}{10}$.
	}
\end{ex}
%Câu 32...........................
\begin{ex}%[0D3B2-4]%[Dự án đề kiểm tra HKII NH22-23- Nguyễn Cường]%[THPT Bảo Thắng Số 2-Lào Cai]
	Một lô hàng gồm $1000$ sản phẩm, trong đó có $50$ phế phẩm. Lấy ngẫu nhiên từ lô hàng đó $1$ sản
	phẩm. Xác suất để lấy được sản phẩm tốt là?
	\choice
	{$0{,}94$}
	{$0{,}96$}
	{\True $0{,}95$}
	{$0{,}97$}
	\loigiai{
		Ta có $n(\Omega)=\mathrm{C}_{1000}^1=1000$.\\
		Biến cố $A$ \lq\lq Lấy được sản phẩm tốt\rq\rq.\\
		Suy ra, $n(A)=\mathrm{C}_{950}^1=950$.\\
		Vậy xác suất biến cố $A$ là $\mathrm{P}(A)=\dfrac{n(A)}{n(\Omega)}=\dfrac{950}{1000}=0{,}95$.
	}
\end{ex}
%Câu 35...........................
\begin{ex}%[0D3B2-3]%[Dự án đề kiểm tra HKII NH22-23- Nguyễn Cường]%[THPT Bảo Thắng Số 2-Lào Cai]
	Lớp A có $9$ học sinh xuất sắc, lớp B có $10$ học sinh xuất sắc. Chọn ngẫu nhiên $2$ trong các học sinh
	đó đi dự lễ tuyên dương học sinh tiêu biểu. Xác suất để $2$ học sinh được chọn từ cùng một lớp là?
	\choice
	{$\dfrac{10}{19}$}
	{\True $\dfrac{9}{19}$}
	{$\dfrac{13}{18}$}
	{$\dfrac{11}{18}$}
	\loigiai{
		Ta có $n(\Omega)=\mathrm{C}_{19}^2=171$.\\
		Biến cố $A$ \lq\lq$2$ học sinh được chọn từ cùng một lớp\rq\rq.\\
		Suy ra, $n(A)=\mathrm{C}_{10}^2+\mathrm{C}_9^2=81$.\\
		Vậy xác suất biến cố $A$ là $\mathrm{P}(A)=\dfrac{n(A)}{n(\Omega)}=\dfrac{81}{171}=\dfrac{9}{19}$.
	}
\end{ex}

\Closesolutionfile{ans}
%\begin{center}
%	\textbf{ĐÁP ÁN}
%	\inputansbox{10}{ans/ans}	
%\end{center}
\begin{center}
	\textbf{PHẦN 2 - TỰ LUẬN}
\end{center}
\begin{bt}%[0D2K2-5]%[Dự án đề kiểm tra HKII NH22-23- Nguyễn Cường]%[THPT Bảo Thắng Số 2-Lào Cai]
Từ các chữ số $1$; $2$; $3$; $4$; $5$; $6$; $7$; $8$.
	\begin{enumerate}
		\item Có thể lập được bao nhiêu số tự nhiên có ba chữ số khác nhau và chia hết cho $3$?
		\item Có thể lập được bao nhiêu số tự nhiên chẵn có bốn chữ số khác nhau?
	\end{enumerate}
\loigiai{
\begin{enumerate}
	\item Từ các chữ số đã cho, ta phân chia thành ba tập hợp như sau
	\begin{itemize}
		\item Tập hợp các chữ số chia hết cho $3$ có $A=\{3;6\}$.
		\item Tập hợp các chữ số chia $3$ dư $1$ có $B=\{1;4;7\}$.
		\item Tập hợp các chữ số chia $3$ dư $2$ có $C=\{2;5;8\}$.
	\end{itemize}
Lập các bộ ba mà tổng của chúng chia hết cho $3$ có các khả năng.
\begin{itemize}
	\item $3$ chữ số này đều thuộc tập hợp $B$ hoặc đều thuộc tập hợp $C$ nên có $1+1=2$ bộ.
	\item $3$ chữ số này mỗi chữ số lần lượt thuộc tập $A$, $B$, $C$ nên có $3\cdot 3\cdot 2=18$ bộ.
\end{itemize}
Suy ra có $20$ bộ ba số mà tổng của chúng chia hết cho $3$.\\
Mỗi bộ số vậy sẽ lập được $3!=6$ số có ba chữ số khác nhau và chia hết cho $3$.\\
Vậy, có $20\cdot 6=120$ số thỏa mãn yêu cầu bài toán.
	\item Số tự nhiên chẵn có bốn chữ số khác nhau là $\overline{abcd}$.
	\begin{itemize}
		\item Chọn chữ số $d$ có $4$ cách chọn do $d\in\{2;4;6;8\}$.
		\item Chọn chữ số $a$ có $7$ cách do $a\ne d$.
		\item Chọn chữ số $b$ có $6$ cách do $b\ne a$, $b\ne d$.
		\item Chọn chữ số $c$ có $5$ cách do $c\ne a$, $c\ne b$, $c\ne d$.
	\end{itemize}
Do đó, có $4\cdot 7\cdot 6\cdot 5=840$ số thỏa mãn yêu cầu bài toán.
\end{enumerate}
}
\end{bt}
\begin{bt}%[0D3B2-3]%[Dự án đề kiểm tra HKII NH22-23- Nguyễn Cường]%[THPT Bảo Thắng Số 2-Lào Cai]
	Lớp 10A8 có $10$ bạn nữ và $20$ bạn nam. Thầy giáo chủ nhiệm cần chọn ra $5$ bạn để biên tập Video giới thiệu về lớp mình. Tính xác suất để thầy có thể chọn được
	\begin{enumerate}
		\item $2$ bạn nam và $3$ bạn nữ.
		\item Có nhiều nhất $1$ bạn nam.
	\end{enumerate}
	\loigiai{
		Ta có $n(\Omega)=\mathrm{C}_{30}^5$.
		\begin{enumerate}
			\item Gọi $A$ là biến cố \lq\lq Chọn được $2$ bạn nam và $3$ bạn nữ\rq\rq.
			Ta có $n(A)=\mathrm{C}_{20}^2\cdot\mathrm{C}_{10}^3$.\\
			Khi đó, xác suất của biến cố $A$ là $\mathrm{P}(A)=\dfrac{n(A)}{n(\Omega)}=\dfrac{\mathrm{C}_{20}^2\cdot\mathrm{C}_{10}^3}{\mathrm{C}_{30}^5}=\dfrac{3800}{23751}$.
			\item Gọi $B$ là biến cố \lq\lq Có nhiều nhất $1$ bạn nam\rq\rq.
			\begin{itemize}
				\item Chọn $5$ bạn nữ có $\mathrm{C}_{10}^5$ cách.
				\item Chọn $1$ bạn nam và $4$ bạn nữ có $\mathrm{C}_{20}^1\cdot\mathrm{C}_{10}^4$ cách.
			\end{itemize}
			Ta có $n(B)=\mathrm{C}_{10}^5+\mathrm{C}_{20}^1\cdot\mathrm{C}_{10}^4=4452$.\\
			Khi đó, xác suất của biến cố $B$ là $\mathrm{P}(B)=\dfrac{n(B)}{n(\Omega)}=\dfrac{4452}{\mathrm{C}_{30}^5}=\dfrac{106}{3393}$.
		\end{enumerate}
	}
\end{bt}
\begin{bt}%[0D2G2-1]%[Dự án đề kiểm tra HKII NH22-23- Nguyễn Cường]%[THPT Bảo Thắng Số 2-Lào Cai]
	Xếp ngẫu nhiên $2$ học sinh lớp 10A, $3$ học sinh lớp 10B và $5$ học sinh lớp 10C thành một hàng ngang. Hỏi có bao nhiêu cách xếp để không có học sinh nào của cùng một lớp đứng cạnh nhau.
	\loigiai{
		Số phần tử của không gian mẫu là $n(\Omega)=10!$.\\
		Gọi $A$ là biến cố \lq\lq Không có học sinh nào của cùng một lớp đứng cạnh nhau\rq\rq.\\
		Sắp xếp $5$ học sinh lớp 10C có $5!$ cách.\\
		Ứng với mỗi cách sắp xếp $5$ học sinh lớp 10C sẽ có $6$ khoảng trống gồm $4$ vị trí ở giữa và hai vị trí ở hai đầu để xếp các học sinh còn lại.
	\begin{enumerate}
		\item Trường hợp 1. Xếp $3$ học sinh 10B vào $4$ vị trí ở giữa (không xếp vào hai đầu), có $\mathrm{A}_4^3$ cách.\\
		Ứng với mỗi cách xếp đó, chọn $1$ trong $2$ học sinh lớp 10A xếp vào $4$ vị trí trống để hai học sinh 10C không được cạnh nhau có $2$ cách.\\
		Học sinh 10A còn lại còn có $8$ vị trí để sắp xếp.\\
		Theo quy tắc nhân ta có $5!\cdot \mathrm{A}_4^3\cdot 2\cdot 8$ cách.
		\item Xếp $2$ trong $3$ học sinh lớp 10B vào $4$ vị trí trống ở giữa và học sinh còn lịa xếp vào $2$ cầu có $\mathrm{A}_4^2\cdot 2\cdot \mathrm{C}_2^1$.\\
		Ứng với mỗi cách xếp đó sẽ còn $2$ vị trí ở giữa, xếp hai học sinh lớp 10A vào đó có $2!$ cách.\\
		Theo quy tắc nhân, có $\mathrm{A}_4^2\cdot 2\cdot \mathrm{C}_2^1\cdot 2!$ cách.
	\end{enumerate}
Suy ra $n(A)=5!\cdot \mathrm{A}_4^3\cdot 2\cdot 8+\mathrm{A}_4^2\cdot 2\cdot \mathrm{C}_2^1\cdot 2!=63360$ cách.\\
Vậy $\mathrm{P}(A)=\dfrac{n(A)}{n(\Omega)}=\dfrac{63360}{10!}=\dfrac{11}{630}$.
	}
\end{bt}