
\de{ĐỀ THI HỌC KỲ II NĂM HỌC 2022-2023}{THPT Trần Phú - Hà Nội}
\begin{center}
	\textbf{PHẦN 1 - TRẮC NGHIỆM}
\end{center}
\Opensolutionfile{ans}[ans/ans]
%Câu1
\begin{ex}%[0X2B2-2]%[Dự án đề kiểm tra HKII NH22-23-Phan Trung Hiếu]%[THPT Trần Phú - Hà Nội]
	Một nhóm có $7$ nam $6$ nữ. Chọn ra $3$ người sao cho trong đó có ít nhất $1$ nữ. Hỏi có bao nhiêu cách?
	\choice
	{$300$}
	{$250$}
	{\True $251$}
	{$252$}
	\loigiai{
		Số cách để chọn ra 3 người sao cho trong đó có ít nhất $1$ nữ là $6\cdot\mathrm{C}_7^2 = 251$.
	}
\end{ex}
%Câu2
\begin{ex}%[0H4Y2-1]%[Dự án đề kiểm tra HKII NH22-23-Phan Trung Hiếu]%[THPT Trần Phú - Hà Nội]
	Đường tròn $(C)\colon x^2+y^2-x+y-1=0$ có tâm $I$ và bán kính $R$ là
	\choice
	{$I(-1;1)$, $R=1$}
	{$I(1;-1)$, $R=\sqrt{6}$}
	{$I\left(-\dfrac{1}{2};\dfrac{1}{2}\right)$, $R =\dfrac{\sqrt{6}}{2}$}
	{\True $I\left(\dfrac{1}{2};-\dfrac{1}{2}\right)$, $R =\dfrac{\sqrt{6}}{2}$}
	\loigiai{
		Đường tròn $(C)\colon x^2+y^2-x+y-1=0$ có tâm $I$ và bán kính $R$ là
		\begin{itemize}
			\item $I\left(\dfrac{1}{2};-\dfrac{1}{2}\right)$.
			\item $R= \sqrt{\left(\dfrac{1}{2}\right)^2+\left(-\dfrac{1}{2}\right)^2-(-1)}=\dfrac{\sqrt{6}}{2}$.
		\end{itemize} 
	}
\end{ex}
%Câu3
\begin{ex}%[0X2B2-1]%[Dự án đề kiểm tra HKII NH22-23-Phan Trung Hiếu]%[THPT Trần Phú - Hà Nội]
	Một tổ có $5$ học sinh nam và $5$ học sinh nữ xếp thành một hàng dọc sao cho không có học sinh cùng giới tính đứng kề nhau. Số cách sắp xếp là
	\choice
	{\True $2\cdot(5!)^2$}
	{$(5!)^2$}
	{$10!$}
	{$2\cdot 5!^2$}
	\loigiai{
		\begin{itemize}
			\item \textbf{Bước 1:} Xếp $5$ học sinh nam có $5!$ cách.
			\item \textbf{Bước 2:} Xếp $5$ học sinh nữ có $5!$ cách.
			\item \textbf{Bước 3:} Hoán vị $5$ nam và $5$ nữ có $2!$ cách.
		\end{itemize}
		Theo quy tắc nhân, có tất cả $2\cdot(5!)^2$ cách.
	}
\end{ex}
%Câu4
\begin{ex}%[0H3Y1-3]%[Dự án đề kiểm tra HKII NH22-23-Phan Trung Hiếu]%[THPT Trần Phú - Hà Nội]
	Trong mặt phẳng tọa độ $Oxy$, cho $A(5;3)$, $B(7;8)$. Tìm tọa độ của véctơ $\vv{AB}$
	\choice
	{$(15;10)$}
	{$(2;6)$}
	{\True $(2;5)$}
	{$(-2;-5)$}
	\loigiai{
		Tọa độ của véctơ $\vv{AB}=(7-5;8-3) = (2;5)$.
	}
\end{ex}
%Câu5
\begin{ex}%[0X2B3-2]%[Dự án đề kiểm tra HKII NH22-23-Phan Trung Hiếu]%[THPT Trần Phú - Hà Nội]
	Số hạng chứa $x^2y^2$ trong khai triển $(3x+2y)^4$ là
	\choice
	{$\mathrm{C}_4^2x^2y^2$}
	{\True $6(3x)^2(2y)^2$}
	{$6\mathrm{C}_4^2x^2y^2$}
	{$36\mathrm{C}_4^2x^2y^2$}
	\loigiai{
		Ta có
		\begin{equation*}
			(3x+2y)^4=\mathrm{C}_4^0(3x)^4+\mathrm{C}_4^1(3x)^3(2x)^1+\mathrm{C}_4^2(3x)^2(2x)^2+\mathrm{C}_4^3(3x)^1(2x)^3+\mathrm{C}_4^4(2x)^4.
		\end{equation*}
		Số hạng chứa $x^2y^2$ trong khai triển $(3x+2y)^4$ là $\mathrm{C}_4^2(3x)^2(2x)^2 = 6(3x)^2(2y)^2$.
	}
\end{ex}
%Câu6
\begin{ex}%[0H4B3-4]%[Dự án đề kiểm tra HKII NH22-23-Phan Trung Hiếu]%[THPT Trần Phú - Hà Nội]
	Cặp điểm nào là tiêu điểm của Hypebol $\dfrac{x^2}{9}-\dfrac{y^2}{5}=1$?
	\choice
	{$(4;0)$ và $(-4;0)$}
	{$(0;\sqrt{14})$ và $(0;-\sqrt{14})$}
	{$(2;0)$ và $(-2;0)$}
	{\True $(\sqrt{14};0)$ và $(-\sqrt{14};0)$}
	\loigiai{
		Ta có $a=3$, $b=\sqrt{5}$, $c=\sqrt{3^2+(\sqrt{5})^2}=\sqrt{14}$.\\
		Các tiêu điểm của Hypebol là $F_1(\sqrt{14};0)$ và $F_2(-\sqrt{14};0)$
	}
\end{ex}
%Câu7
\begin{ex}%[0X1Y1-2]%[Dự án đề kiểm tra HKII NH22-23-Phan Trung Hiếu]%[THPT Trần Phú - Hà Nội]
	Cho số gần đúng $a = 2851275$ với độ chính xác $d = 400$. Hãy viết quy tròn số $a$.
	\choice
	{$2850025$}
	{$2851575$}
	{\True $2851000$}
	{$2851200$}
	\loigiai{
		Vì độ chính xác đến hàng trăm nên ta quy tròn $a$ đến hàng nghìn, vậy số quy tròn của $a$ là $2851000$.
	}
\end{ex}
%Câu8
\begin{ex}%[0X3Y2-1]%[Dự án đề kiểm tra HKII NH22-23-Phan Trung Hiếu]%[THPT Trần Phú - Hà Nội]
	Gieo một đồng xu cân đối đồng chất hai lần. Số phần tử của biến cố: \lq\lq~mặt sấp xuất hiện đúng $1$ lần\rq\rq~là
	\choice
	{\True $2$}
	{$4$}
	{$5$}
	{$6$}
	\loigiai{
		Gọi $A$ là biến cố: \lq\lq~mặt sấp xuất hiện đúng $1$ lần\rq\rq.\\
		Ta có $A=\{SN;NS\}$, suy ra $n(A)=2$.
	}
\end{ex}
%Câu9
\begin{ex}%[0X2Y3-2]%[Dự án đề kiểm tra HKII NH22-23-Phan Trung Hiếu]%[THPT Trần Phú - Hà Nội]
	Trong khai triển $(2x-1)^5$ có số các số hạng là
	\choice
	{$7$}
	{$5$}
	{$4$}
	{\True $6$}
	\loigiai{
		Số các số hạng là $5+1=6$ số hạng.
	}
\end{ex}
%Câu10
\begin{ex}%[0X3K2-5]%[Dự án đề kiểm tra HKII NH22-23-Phan Trung Hiếu]%[THPT Trần Phú - Hà Nội]
	Cho đa giác đều $20$ đỉnh nội tiếp đường tròn tâm $O$ . Chọn ngẫu nhiên $3$ đỉnh của đa giác đó. Tính xác suất để $3$ đỉnh được chọn tạo thành một tam giác không có cạnh nào là cạnh của đa giác đã cho
	\choice
	{$\dfrac{34}{49}$}
	{$\dfrac{1033}{1140}$}
	{\True $\dfrac{40}{57}$}
	{$\dfrac{41}{57}$}
	\loigiai{
		Số phần tử của không gian mẫu là $n(\Omega) = \mathrm{C}_{20}^3$.\\
		Gọi $A$ là biến cố \lq\lq~$3$ đỉnh được chọn tạo thành một tam giác không có cạnh nào là cạnh của đa giác đã cho\rq\rq.\\
		Suy ra, biến cố $\overline{A}$ là \lq\lq~chọn được ba đỉnh tạo thành tam giác có một cạnh hoặc hai cạnh là cạnh của đa giác đã cho\rq\rq.\\
		\begin{itemize}
			\item \textbf{Truờng hợp 1:} Chọn tam giác có $2$ cạnh là $2$ cạnh của đa giác đã cho, có tất cả $20$ cách.
			\item \textbf{Trường hợp 2:} Chọn tam giác có đúng $1$ cạnh là cạnh của đa giác đã cho, ta chọn ra $1$ cạnh và $1$ đỉnh không liền kề với $2$ đỉnh của cạnh đó.\\
			Khi đó, có tất cả $20$ cách chọn $1$ cạnh và $16$ cách chọn đỉnh.\\
			Theo quy tắc nhân, có tất cả $20\cdot16$ cách chọn.
 		\end{itemize}
 		Từ hai trường hợp trên, theo quy tắc cộng, có tất cả $20+20\cdot16 = 340$ cách.\\
 		Xác suất để $3$ đỉnh được chọn tạo thành một tam giác không có cạnh nào là cạnh của đa giác đã cho là
 		\begin{equation*}
 			P(A) = 1-P(\overline{A}) = 1 - \dfrac{n(\overline{A})}{n(\Omega)} = 1-\dfrac{340}{C_{20}^3}=\dfrac{40}{57}.
 		\end{equation*}
	}
\end{ex}
%Câu11
\begin{ex}%[0X3Y2-1]%[Dự án đề kiểm tra HKII NH22-23-Phan Trung Hiếu]%[THPT Trần Phú - Hà Nội]
	Gieo một con súc sắc cân đối đồng chất hai lần. Gọi $A$ là biến cố: \lq\lq Tích số chấm trong hai lần gieo bằng 6\rq\rq.Tập hợp các phần tử của biến cố $A$ là
	\choice
	{$\{(6;1),(3;2)\}$}
	{$\{(1;5),(2;4),(3;3)\}$}
	{$\{(1;6),(2;3)\}$}
	{\True $\{(1;6),(6;1),(2;3),(3;2)\}$}
	\loigiai{
		Tập hợp các phần tử của biến cố $A$ là: $A=\{(1;6),(6;1),(2;3),(3;2)\}$.
	}
\end{ex}
%Câu12
\begin{ex}%[0X3B2-6]%[Dự án đề kiểm tra HKII NH22-23-Phan Trung Hiếu]%[THPT Trần Phú - Hà Nội]
	Chọn ngẫu nhiên một số tự nhiên có hai chữ số. Xác suất để chọn được một số lẻ và chia hết cho $9$ là
	\choice
	{$\dfrac{1}{9}$}
	{\True $\dfrac{1}{18}$}
	{$\dfrac{1}{12}$}
	{$\dfrac{1}{10}$}
	\loigiai{
		Gọi số tự nhiên có hai chữ số là $X=\overline{ab}$.\\
		Gọi $A$ là biến cố: \lq\lq chọn được một số lẻ và chia hết cho $9$\rq\rq.\\ 
		Không gian mẫu $n(\Omega) = 90$.\\
		Số lẻ và chia hết cho $9$ là $27$, $45$, $63$, $81$, $99$ nên có tất cả $4$ cách.\\
		Xác suất để chọn được một số lẻ và chia hết cho $9$ là
		\begin{equation*}
			P(A) = \dfrac{n(A)}{n(\Omega)} = \dfrac{4}{90} = \dfrac{1}{18}.
		\end{equation*}
	}
\end{ex}
%Câu13
\begin{ex}%[0X2Y1-1]%[Dự án đề kiểm tra HKII NH22-23-Phan Trung Hiếu]%[THPT Trần Phú - Hà Nội]
	Có $5$ quyển sách Tiếng Anh khác nhau, $6$ quyển sách Toán khác nhau và $8$ quyển sách Văn khác nhau. Số cách chọn $1$ quyển sách là
	\choice
	{$8$}
	{$240$}
	{$6$}
	{\True $19$}
	\loigiai{
		Số cách chọn $1$ quyển sách là $5+6+8 = 19$ cách.
	}
\end{ex}
%Câu14
\begin{ex}%[0X1B3-1]%[Dự án đề kiểm tra HKII NH22-23-Phan Trung Hiếu]%[THPT Trần Phú - Hà Nội]
	Điểm kiểm tra môn Toán cuối năm của một nhóm gồm $9$ học sinh lớp $10$ lần lượt là $1$; $1$; $3$; $6$; $7$; $8$; $8$; $9$; $10$. Điểm trung bình của cả nhóm gần nhất với số nào dưới đây?
	\choice
	{$7{,}5$}
	{$7$}
	{$6{,}5$}
	{\True $5{,}9$}
	\loigiai{
		Điểm trung bình của cả nhóm là
		\begin{equation*}
			\overline{X} = \dfrac{1+3+3+6+7+8+8+9+10}{9} \approx 5{,}9.
		\end{equation*}
	}
\end{ex}
%Câu15
\begin{ex}%[0H4B1-2]%[Dự án đề kiểm tra HKII NH22-23-Phan Trung Hiếu]%[THPT Trần Phú - Hà Nội]
	Đường thẳng đi qua $A(-1;2)$ nhận $\vv{n}=(2;4)$ làm véctơ pháp tuyến có phương trình là
	\choice
	{$x-2y-4=0$}
	{$x+y+4=0$}
	{\True $x-2y+5=0$}
	{$-x+2y-4=0$}
	\loigiai{
		Phương trình đường thẳng đi qua $A(-1;2)$ nhận $\vv{n}=(2;-4)$ làm véctơ pháp tuyến là
		\begin{equation*}
			2(x+1) - 4(y-2)=0\Leftrightarrow 2x-4y +10=0\Leftrightarrow x-2y+5=0.
		\end{equation*}
	}
\end{ex}
%Câu16
\begin{ex}%[0H4B3-2]%[Dự án đề kiểm tra HKII NH22-23-Phan Trung Hiếu]%[THPT Trần Phú - Hà Nội]
	Viết phương trình chính tắc của elip $(E)$ biết trục lớn $2a=8$, trục bé $2b=6$.
	\choice
	{\True $(E)\colon\dfrac{x^2}{16}+\dfrac{y^2}{9}=1$}
	{$(E)\colon\dfrac{x^2}{25}+\dfrac{y^2}{9}=1$}
	{$(E)\colon\dfrac{x^2}{25}+\dfrac{y^2}{16}=1$}
	{$(E)\colon\dfrac{x^2}{9}+\dfrac{y^2}{16}=1$}
	\loigiai{
		Ta có $a=4$, $b=3$.\\
		Phương trình chính tắc của elip $(E)$ là
		\begin{equation*}
			\dfrac{x^2}{a^2}+\dfrac{y^2}{b^2}=1\Leftrightarrow\dfrac{x^2}{16}+\dfrac{y^2}{9}=1.
		\end{equation*}
	}
\end{ex}
%Câu17
\begin{ex}%[0H4Y1-1]%[Dự án đề kiểm tra HKII NH22-23-Phan Trung Hiếu]%[THPT Trần Phú - Hà Nội]
	Cho đường thẳng $\Delta\colon x-3y-2=0$. Véctơ nào không phải là véctơ pháp tuyến của $\Delta$.
	\choice
	{$(1;-3)$}
	{\True $(3;1)$}
	{$\left(\dfrac{1}{3};-1\right)$}
	{$(-2;6)$}
	\loigiai{
		Véctơ không phải là véctơ pháp tuyến của $\Delta$ là $\vv{n} = (3;1)$.
	}
\end{ex}
%Câu18
\begin{ex}%[0X3B2-1]%[Dự án đề kiểm tra HKII NH22-23-Phan Trung Hiếu]%[THPT Trần Phú - Hà Nội]
	Gieo một con súc sắc cân đối đồng chất hai lần. Xác suất của biến cố: \lq\lq Tổng số chấm trong hai lần gieo không vượt quá $10$\rq\rq~là:
	\choice
	{$\dfrac{1}{12}$}
	{$\dfrac{5}{6}$}
	{$\dfrac{1}{6}$}
	{\True $\dfrac{11}{12}$}
	\loigiai{
		Ta có $n(\Omega)=36$.\\
		Gọi $A$ là biến cố \lq\lq Tổng số chấm trong hai lần gieo không vượt quá $10$\rq\rq.\\
		Khi đó, $n(A)=4+5=6+6+6+6 =33$.\\
		Xác xuất của biến cố $A$ là
		\begin{equation*}
			P(A) = \dfrac{n(A)}{n(\Omega)} = \dfrac{33}{36} =\dfrac{11}{12}.
		\end{equation*}
	}
\end{ex}
%Câu19
\begin{ex}%[0H4B1-2]%[Dự án đề kiểm tra HKII NH22-23-Phan Trung Hiếu]%[THPT Trần Phú - Hà Nội]
	Cho hình hành $ABCD$, biết $A(-2;1)$ và phương trình đường thẳng $CD$ là $3x-4y-5=0$. Phương trình tham số của đường thẳng $AB$ là
	\choice
	{\True $\heva{&x=-2-4t\\&y=1-3t}$}
	{$\heva{&x=-2+3t\\&y=-2-2t}$}
	{$\heva{&x=-2-3t\\&y=1-4t}$}
	{$\heva{&x=-2-3t\\&y=1+4t}$}
	\loigiai{
		Vì $ABCD$ là hình bình hành nên $AB$ song song với $CD$.\\
		Do đó, véctơ pháp tuyến của đường thẳng $CD$ cũng là véctơ pháp tuyến của phương trình đường thẳng $AB$.\\
		Vậy $n_{\vv{AB}} = (3;-4)$.\\
		Suy ra, véctơ chỉ phương của phương trình đường thẳng $AB$ là $u_{\vv{AB}} = (4;3)=(-4;-3)$.\\
		Khi đó, phương trình tham số của đường thẳng $AB$ là $\heva{&x=-2-4t\\&y=1-3t}$.
	}
\end{ex}
%Câu20
\begin{ex}%[0X1B3-3]%[Dự án đề kiểm tra HKII NH22-23-Phan Trung Hiếu]%[THPT Trần Phú - Hà Nội]
	Tìm tứ phân vị của mẫu số liệu sau
	\begin{center}
	\begin{tabular}{|c|c|c|c|c|c|c|c|c|c|c|c|}
		\hline
		12&3&6&15&27&33&31&18&29&54&1&8\\
		\hline
	\end{tabular}
	\end{center}
	\choice
	{$Q_1=7$, $Q_2=17{,}5$, $Q_3=30$}
	{\True $Q_1=7$, $Q_2=16{,}5$, $Q_3=30$}
	{$Q_1=7$, $Q_2=16$, $Q_3=30{,}5$}
	{$Q_1=7{,}5$, $Q_2=16{,}5$, $Q_3=30$}
	\loigiai{
		Sắp xếp mẫu số liệu theo thứ tự không giảm
		\begin{equation*}
			1,3,6,8,12,15,18,27,29,31,33,54
		\end{equation*}
		Ta có $Q_2=\dfrac{15+18}{2} = 16{,}5$, $Q_1 = \dfrac{6+8}{2}=7$, $Q_3 = \dfrac{29+31}{2}=30$.
	}
\end{ex}
%Câu21
\begin{ex}%[0H4B1-1]%[Dự án đề kiểm tra HKII NH22-23-Phan Trung Hiếu]%[THPT Trần Phú - Hà Nội]
	Cho tam giác $ABC$ có tọa độ ba đỉnh lần lượt là $(2;3)$, $(5;4)$, $(2;2)$. Tọa độ trọng tâm $G$ của tam giác có tọa độ là
	\choice
	{$(1;1)$}
	{$(2;2)$}
	{\True $(3;3)$}
	{$(4;4)$}
	\loigiai{
		Tọa độ trọng tâm $G$ của tam giác có tọa độ là
		\begin{equation*}
			\heva{&x_G = \dfrac{x_A+x_B+x_C}{3}=\dfrac{2+5+2}{3}=3\\&y_G = \dfrac{y_A+y_B+y_C}{3}=\dfrac{3+4+2}{3}=3.}
		\end{equation*}
	}
\end{ex}
%Câu22
\begin{ex}%[0H4B1-5]%[Dự án đề kiểm tra HKII NH22-23-Phan Trung Hiếu]%[THPT Trần Phú - Hà Nội]
	Khoảng cách từ điểm $M(5;-1)$ đến đường thẳng $\Delta\colon 3x+2y+13=0$ là
	\choice
	{\True $2\sqrt{13}$}
	{$2$}
	{$\dfrac{28}{\sqrt{13}}$}
	{$\dfrac{13}{\sqrt{2}}$}
	\loigiai{
		Khoảng cách từ điểm $M(5;-1)$ đến đường thẳng $\Delta\colon 3x+2y+13=0$ là
		\begin{equation*}
			d(M,\Delta) = \dfrac{|3\cdot5-2\cdot1+13|}{\sqrt{3^2+2^2}} = 2\sqrt{13}.
		\end{equation*}
	}
\end{ex}
%Câu23
\begin{ex}%[0X2B2-4]%[Dự án đề kiểm tra HKII NH22-23-Phan Trung Hiếu]%[THPT Trần Phú - Hà Nội]
	Số tam giác xác định bởi các đỉnh của một đa giác đều $10$ cạnh là
	\choice
	{$35$}
	{\True $120$}
	{$240$}
	{$710$}
	\loigiai{
		Số tam giác xác định bởi các đỉnh của một đa giác đều 10 cạnh là $C_{10}^3 = 120$.
	}
\end{ex}
%Câu24
\begin{ex}%[0X3B2-2]%[Dự án đề kiểm tra HKII NH22-23-Phan Trung Hiếu]%[THPT Trần Phú - Hà Nội]
	Hai bạn lớp $A$ và hai bạn lớp $B$ được xếp vào $4$ ghế hàng ngang. Xác suất sao cho các bạn cùng lớp ngồi cạnh nhau bằng
	\choice
	{$\dfrac{1}{2}$}
	{$\dfrac{1}{4}$}
	{$\dfrac{1}{3}$}
	{\True $\dfrac{2}{3}$}
	\loigiai{
		Ta có $n(\Omega) = 4!$.\\
		Gọi $A$ là biến cố \lq\lq các bạn cùng lớp ngồi cạnh nhau\rq\rq.\\
		Biến cố $\overline{A}$:\lq\lq các bạn cùng lớp không ngồi cạnh nhau\rq\rq.\\
		Khi đó, $n(\overline{A}) = 2\cdot2\cdot2=8$ cách.\\
		Xác suất sao cho các bạn cùng lớp ngồi cạnh nhau là
		\begin{equation*}
			1-P(\overline{A})=1-\dfrac{n(\overline{A})}{n(\Omega)} = 1- \dfrac{8}{4!} = \dfrac{2}{3}.
		\end{equation*}
	}
\end{ex}
%Câu25
\begin{ex}%[0X3B2-4]%[Dự án đề kiểm tra HKII NH22-23-Phan Trung Hiếu]%[THPT Trần Phú - Hà Nội]
	Một hộp chứa $11$ quả cầu gồm $5$ quả màu xanh và $6$ quả màu đỏ. Chọn ngẫu nhiên đồng thời $2$ quả cầu từ hộp đó. Xác suất để $2$ quả cầu chọn ra cùng màu bằng
	\choice
	{$\dfrac{8}{11}$}
	{$\dfrac{5}{22}$}
	{$\dfrac{6}{11}$}
	{\True $\dfrac{5}{11}$}
	\loigiai{
		Ta có $n(\Omega)=C_{11}^2$.\\
		Gọi $A$ là biến cố \lq\lq $2$ quả cầu chọn ra cùng màu\rq\rq.\\
		Khi đó, $n(A) = C_5^2+C_6^2$.\\
		Xác suất để $2$ quả cầu chọn ra cùng màu là
		\begin{equation*}
			P(A)=\dfrac{n(A)}{n(\Omega)} = \dfrac{C_5^2+C_6^2}{C_{11}^2}=\dfrac{5}{11}.
		\end{equation*}
	}
\end{ex}

\Closesolutionfile{ans}
%\begin{center}
%	\textbf{ĐÁP ÁN}
%	\inputansbox{10}{ans/ans}	
%\end{center}
\begin{center}
	\textbf{PHẦN 2 - TỰ LUẬN}
\end{center}

\begin{bt}%[0X1B3-2]%[Dự án đề kiểm tra HKII NH22-23-Phan Trung Hiếu]%[THPT Trần Phú - Hà Nội]
	Thống kê điểm kiểm tra môn Toán của 45 học sinh lớp 10A1 như sau
	\begin{center}
		\begin{tabular}{|c|c|c|c|c|c|c|}
			\hline
			Điểm&5&6&7&8&9&10\\
			\hline
			Số học sinh&2&11&9&16&4&3\\
			\hline
		\end{tabular}
	\end{center}
	Tính số trung vị trong các bài kiểm tra đó.
\loigiai{
Lớp có $45$ hs nên số trung vị là số ở vị trí thứ $23$. Đó là $8$ điểm.
}
\end{bt}
\begin{bt}%[0X3K2-1]%[Dự án đề kiểm tra HKII NH22-23-Phan Trung Hiếu]%[THPT Trần Phú - Hà Nội]
	Gieo một đồng xu cân đối đồng chất liên tiếp $3$ lần. Gọi $A$ là biến cố \lq\lq trong $3$ lần gieo có đúng một lần xuất hiện mặt sấp\rq\rq. Tính xác suất của biến cố $A$.
\loigiai{
	Số phần tử của không gian mẫu là $n(\Omega)=2^3=8$.\\
	Tập các phần tử thuận lợi cho biến cố $A$ là $\{SNN, NSN, NNS\}$.\\
	Số phần tử của không gian biến cố $A$ là $n(A)=3$.\\
	Xác suất của biến cố $A$ là $P(A)=\dfrac{n(A)}{n(\Omega)} = \dfrac{3}{8}$.
}
\end{bt}
\begin{bt}%[0X3G2-6]%[Dự án đề kiểm tra HKII NH22-23-Phan Trung Hiếu]%[THPT Trần Phú - Hà Nội]
	Gọi $S$ là tập hợp các số tự nhiên có $3$ chữ số đôi một khác nhau được lập thành từ các chữ số $0$, $1$, $2$, $3$, $4$, $5$, $6$. Chọn ngẫu nhiên một số từ $S$, tính xác suất để số được chọn chia hết cho $5$.
\loigiai{
	Số phần tử của $S$ là $5\cdot A_6^2=180$.\\
	Không gian mẫu là chọn ngẫu nhiên $1$ số từ tập $S$.\\
	Suy ra số phần tử của không gian mẫu là $n(\Omega)=C_{180}^1=180$.\\
	Gọi $A$ là biến cố \lq\lq Số được chọn chia hết cho 5\rq\rq.\\
	Lập số có $3$ chữ số khác nhau chia hết cho $5$ từ các chữ số $0$, $1$, $2$, $3$, $4$,  $5$, $6$.\\
	\textbf{Trường hợp 1:} Chữ số tận cùng bằng $0$, lập được $A_6^2= 30$ số.\\
	\textbf{Trường hợp 2:} Chữ số tận cùng bằng $5$, lập được $5\cdot5=25$ số.\\
	Suy ra số phần tử của biến cố $A$ là $n(A)=30+25=55$.\\
	Vậy xác suất cần tìm là
	\begin{equation*}
		P(A)=\dfrac{n(A)}{n(\Omega)} = \dfrac{55}{180} = \dfrac{11}{36}.
	\end{equation*}
}
\end{bt}
\begin{bt}%[0X3G2-4]%[Dự án đề kiểm tra HKII NH22-23-Phan Trung Hiếu]%[THPT Trần Phú - Hà Nội]
	Một chiếc hộp đựng $8$ viên bi màu xanh được đánh số từ $1$ đến $8$, $9$ viên bi màu đỏ được đánh số từ $1$ đến $9$ và $10$ viên bi màu vàng được đánh số từ $1$ đến $10$. Một người chọn ngẫu nhiên $3$ viên bi trong hộp. Tính xác suất để $3$ viên bi được chọn có số đôi một khác nhau.
\loigiai{
	Số phân tử của không gian mẫu là $n(\Omega)=C_{27}^3=2925$.\\
	Để đếm số phần tử của không gian thuận lợi cho biến cố $A$ trong bài ta chia nhiều trường hợp theo số màu của $3$ viên bi được chọn.\\
	\textbf{Trường hợp 1:} Một màu.\\
	Có $C_8^3+C_9^3+C_{10}^3 = 260$ phần tử (tương ứng với màu xanh, đỏ, vàng).\\
	\textbf{Trường hợp 2:} Hai màu.\\
	Có $C_8^1\cdot C_8^2 + C_8^2\cdot C_7^1+C_8^1\cdot C_9^2 + C_8^2\cdot C_8^1+C_9^1\cdot C_9^2 + C_9^2\cdot C_8^1=1544$ phần tử (ứng với các cặp màu xanh-đỏ, đỏ-vàng, xanh-vàng).\\
	\textbf{Trường hợp 3:} Ba màu.\\
	 Có $C_8^1\cdot C_8^1\cdot C_8^1=512$ phần tử (tương ứng với màu xanh, đỏ, vàng).\\
	 Như vậy, $n(A)=2316$.\\
	 Vậy xác suất của biến cố $A$ là $P(A) = \dfrac{n(A)}{n(\Omega)} = \dfrac{2316}{2925}=\dfrac{772}{975}$.
}
\end{bt}
\begin{bt}%[0H4B1-2]%[0H4B1-4][0H4B2-2]%[Dự án đề kiểm tra HKII NH22-23-Phan Trung Hiếu]%[THPT Trần Phú - Hà Nội]
	Trong mặt phẳng tọa độ $Oxy$, cho hai điểm $A(-2;1)$, $B(2;4)$ và đường thẳng $d\colon3x+4y-10=0$.
	\begin{enumerate}
		\item Viết phương trình đường thẳng $\Delta$ đi qua hai điểm $A(-2;1)$ và $B(2;4)$.
		\item Tính cosin góc giữa hai đường thẳng $d$ và $\Delta$.
		\item Viết phương trình đường tròn đường kính $AB$.
	\end{enumerate}
\loigiai{
\begin{enumerate}
	\item Đường thẳng $AB$ nhận $\vv{AB}=(4;3)$ làm véctơ chỉ phương, do đó một véctơ pháp tuyến của đuởng thẳng $AB$ là $\vv{n}=(3;-4)$.\\
	Vậy phương trình tổng quát của đường thẳng $AB$ là 
	\begin{equation*}
		3(x+2)-4(y-1) = 0\Leftrightarrow 3x-4y+10=0.
	\end{equation*}
	\item Véctơ pháp tuyến của đường thẳng $d$ là $\vv{n}_d=(3;4)$, $\vv{n}_{\Delta}=(3;-4)$.
	\begin{equation*}
		\cos(d,\Delta) = \left|\cos\left(\vv{n_d},\vv{n}_{\Delta}\right)\right|=\dfrac{\left|\vv{n}_d\cdot\vv{n}_\Delta\right|}{\left|\vv{n}_d\right|\cdot\left|\vv{n}_\Delta\right|}=\dfrac{7}{25}.
	\end{equation*}
	\item Tâm $I\left(0;\dfrac{5}{2}\right)$, $R=\dfrac{5}{2}$.\\
	Phương trình đường tròn đường kính $AB$ là
	\begin{equation*}
		x^2 + \left(y-\dfrac{5}{2}\right)^2=\dfrac{25}{4}.
	\end{equation*}
\end{enumerate}
}
\end{bt}